\documentclass[11pt,english]{scrartcl}
\usepackage[T1]{fontenc}
\usepackage{microtype}
\usepackage[utf8]{inputenc}
\usepackage{babel}
\usepackage{hyperref}
\setlength\parindent{0pt}

\usepackage{url}
\def\UrlFont{\fontfamily{lmtt}\selectfont \footnotesize}

\begin{document}
\pagenumbering{gobble}

\begin{flushleft}
    \large
    \textbf{Re-submission of manuscript:} Reactive SINDy: Discovering governing reactions from concentration data
\end{flushleft}
\begin{flushright}
    \today
\end{flushright}
~\\

Dear editorial board members,\\

please find attached our revised manuscript in tex format and another compiled pdf document highlighting the changes. Further we give the response to the reviewers below.\\

\textbf{{\large Response to Reviewer \#1}}\newline

Thank you for the review.\newline

\textbf{{\large Response to Reviewer \#2}}\newline

Thank you for the detailed review. We have addressed the comments in the
following way\newline

\textbf{Original comment:}
{This paper describes the modification of an approach called SINDY that first appeared in PNAS in 2015. It is used to estimate parameters in a system of chemical reactions in the presence or absence of noise. The method is applied to a rather artificial gene regulatory models in both a deterministic and stochastic setting. I have the following minor comments and one major comment.\\
1. The authors refer to ``ansatz'' functions. I know what ansatz means in German but what is meant here? - please clarify.}\newline

\textbf{Answer:}
The meaning of the word ``ansatz'' is now clarified when it first occurs (page 1).\newline

\textbf{Original comment:}
{2. Bottom of page 3, column 1. The authors mention a suitable choice of hyper parameters. But why is alpha =$1.91\times 10^{-7}$ suitable? And at this stage I still do not know what Kappa represents.\\
3. Page 3, line -5, Col. 2. There is a line that makes no sense.}\newline

\textbf{Answer:}
There was a formatting error which caused certain pieces of
text to disappear and randomly re-appear at a different place in the
document. We hope that now the choice of alpha (cross validation) and
the meaning of kappa (cutoff) are clarified and the line that made no
sense is put back into context.\newline

\textbf{Original comment:}
{My major comment is that the test problem is artificial and boring, with very little interesting dynamics. I would like to see the approach work on a problem that has some bimodality, for example. Furthermore, I would also like to see the method applied to a realistic biological test problem such as the MAPK cascade. Otherwise the impact of the work is quite low. This is not negotiable in terms of acceptance.}\newline

\textbf{Answer:}
We now applied the method to a MAPK cascade as well as a predator-prey model that contains
temporal bimodality and show that we can recover the correct reaction
network with the correct regularization hyperparameters.\newline

Best regards,\\

Moritz Hoffmann, Christoph Fröhner, and Frank Noé

\end{document}
